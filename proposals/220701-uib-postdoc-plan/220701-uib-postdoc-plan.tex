% Postdoc development plan due 2022.07.01.

% vanilla article for latex2rtf
\newif\iflatextortf
\iflatextortf
  \documentclass{article}
  \renewcommand{\cvdoubleitem}[4]{#1: #2; #3: #4}
  \renewcommand{\href}[2]{\htmladdnormallink{\color{blue}\underline{#2}}{#1}}

% else moderncv just looks cool
\else
  \documentclass[11pt,a4paper,sans,colorlinks]{moderncv}
  \name{Julien}{Seguinot}  % only used in metadata
  \moderncvstyle[left]{casual}  % casual (default), classic, oldstyle or banking
  \moderncvcolor{green} % blue (default), orange, green, red, purple, grey or black

\fi

% latex packages
\usepackage[T1]{fontenc}
\usepackage[utf8]{inputenc}
\usepackage[bottom=0.08\paperheight,top=0.08\paperheight]{geometry}

% moderncv section numbers
\newcounter{secnumber}
\renewcommand\sectionstyle[1]{{
  \refstepcounter{secnumber}\sectionfont\textcolor{color1}{\thesecnumber.~#1}}}

% custom commands
\newcommand{\names}[2]{\cvdoubleitem{First name}{#1}{Last name}{#2}}
\newcommand{\guideline}[1]{{\color{color2}\itshape{#1}}}
\newcommand{\todo}[1]{{\color{red}\emph{Todo}: #1}}
% \renewcommand{\guideline}[1]{}

% document properties
\title{Development plan for Postdoctoral Research Fellow}

% ======================================================================
\begin{document}
\setlength{\parskip}{0.5\baselineskip}
% ======================================================================

    \begin{center}
      \textbf{Development plan for Postdoctoral Research Fellow}\\
      Julien~Seguinot \\
      Department of Biological Sciences and
      Bjerknes Centre for Climate Research
    \end{center}

    \guideline{
        Text in italic letters are guidelines and should be deleted once the
        plan is completed.}

    The intention of a postdoctoral research fellow position is, according to
    national regulations, to qualify the candidates to senior academic
    positions. In order to ensure a correct use of these positions, and to
    strengthen career development for postdoctoral research fellows, it is
    mandatory to draw up and submit a development plan within the first 3
    months of the employment period, ref.:
    \href{https://translate.google.com/translate?hl=en&sl=no&u=https://lovdata.no/dokument/SF/forskrift/2006-01-31-102&prev=search&pto=aue}
    {Regulations on terms of employment for positions such as post-doctoral
    fellow, research fellow, research assistant and specialist candidate}.

    The development plan follows the requirements in
    ``\href{https://wiki.uib.no/matnat/images/a/a2/Retningslinjer_for_utviklingsplan_for_Postdoktorstillinger_ved_MN_fakultetet_UiB_-_10.05.-17.docx}
    {Retningslinjene for utviklingsplan for postdoktorstillinger ved Det
    matematisk-naturvitenskapelige fakultet}'', UiB, (and if relevant, the
    requirements from The Research Council of Norway (RCN) as they are outlined
    in the document ``\href{https://www.forskningsradet.no/siteassets/utlysninger/vedlegg-utlysninger/Development-plan-post-doctoral-research-fellows.docx}{Mandatory professional development plan for post-doctoral
    research fellows}'') and reflects the organization of the post doctor period
    and the experiences and qualifications which are to be expected from the
    post doctor during the employment period.


% ----------------------------------------------------------------------
\section{Project description for the research}
% ----------------------------------------------------------------------

    \guideline{
        Planned scientific achievements (=objectives of the research project),
        planned achievements for scientific training (scientific skills). For
        externally financed positions (including RCN-funded projects), refer to
        the project description. The name of the project is given in the
        start-up form.  The project description must be attached to the
        development plan.}

    During my position at the University of Bergen (Apr. 2022 -- Mar. 2025), I
    plan to develop an open-source, automated framework for paleoglacier
    modelling applicable anywhere on Earth. This new tool will offer
    unprecedented accessibility to researchers from other disciplines (e.g.
    geology, ecology), allow for fully automated, consistent, multi-domain
    regional or global paleoglacier studies, and come along a companion
    outreach platform to improve communication on paleoglacier research within
    and beyond academia.

    This work is part of a broader four-year Trond Mohn Foundation (TMS) grant
    to Suzette Flantua on the ``past, present and future of alpine biomes
    worldwide''. Within this project, I contribute to work packages 1 (past) and
    4 (outreach, see proposal) with mountain paleoglacier reconstructions from
    numerical modelling, while two Ph.D. students planned to start in
    Jan. 2023 use my model output to inform their biome connectivity and
    biodiversity work within packages 2 (present) and 3 (future).

    Programming and academic writing have been a main component of my work for
    over a decade. However, by the end of this project, I aim to also become
    fluent in software development, documentation practices, and science
    communication. I am unconvinced whether such skills ``qualify [me] to
    senior academic positions'', but I believe they qualify me to better work.
    More importantly with this project, we will pave the way to a more open and
    more efficient methods and data communication culture in paleoglaciology
    and the rest of academia.


% ----------------------------------------------------------------------
\section{Teaching, supervision and other compulsory duties}
% ----------------------------------------------------------------------

    \guideline{
        (To be evaluated in relation to the department’s needs versus the needs
        of the postdoctoral research fellow). Choose and specify (preferably
        several of) the elements below. Contact the administration in the
        department when planning teaching activities and to obtain information
        about other possible duties/tasks.

        \begin{itemize}
          \item[a.] Teaching, supervision or other work?
          \item[b.] Content, frequency and extent of teaching.
          \item[c.] Supervision ((co-)supervision of master and PhD students).
          \item[d.] Detailed plan for the compulsory work.
          \item[e.] The compulsory work comprises in total x per cent of the position.
        \end{itemize}}

    My prior teaching experience and science background is very different from
    the courses taught within the Department of Biological Sciences where I am
    currently employed, and we need more time to understand if I could teach
    at other departments at the University of Bergen. Thus no formal agreement
    has yet been reached regarding teaching and possible contract extension
    through other duties.

    Nevertheless,
    I am planning to actively attract master students and interns to join our
    project. Unlike lecturing this would be a new experience for me, which I
    see as a good opportunity to learn advising skills and broaden the impact
    of my work. Besides, I will be a co-supervisor for the two Ph.D. students
    on the TMS project, where I hope to contribute with teaching practical
    skills such as academic writing or coding practices, despite having a
    different science background.

% ----------------------------------------------------------------------
\section{Scientific supervision}
% ----------------------------------------------------------------------

    \textbf{a. Responsible for the scientific supervision/guidance of the
            candidate}

    Suzette Flantua

    \textbf{b. Mentor:}

    \guideline{
        (researcher with experience in evaluating applicants for senior
        academic positions).}

    \todo{Still thinking about it...}


% ----------------------------------------------------------------------
\section{Career plan}
% ----------------------------------------------------------------------

    \guideline{
        \begin{itemize}
          \item[a.] The candidate and his/her supervisor must develop a career
            plan in which the postdoctoral position is included
          \item[b.] Future career opportunities both inside and outside of
            academia need to be considered.
        \end{itemize}}

    Despite the ``national regulations'', I do not envision my three-year
    position at the university of Bergen as a platform to ``qualify [...] to
    senior academic positions'' (I also hope that national regulations are
    wrong, as the number of trainees far outweight the number of academic
    positions and thus one could otherwise argue that postdocs like me are a
    vast waste of human resources and public funds). Instead, I see my
    temporary position here as an opportunity to
    continue my work as a researcher, with the freedom to pursue my work in a
    way I believe to be right.

    Ongoing climate change and the COVID-19 pandemic have finished to convince
    me that society needs improved science communication. I believe that not
    only results, but also the research process, need to be more openly
    commicated both within and beyond academia. Thus as a researcher, I am
    dedicated to publishing research methods, software, data, and
    results open access and to communicating science outside academia.

    I see little meaning in pursuing a career in an enviroment favouring mass
    publication of glorified outcomes in a protective intradisciplinary loop.
    Instead, and through this project, I want to contribute to a more open and
    more efficient exchange of scientific methods and method flaws, often doors
    to new research questions, both within and outwards academia. My existing
    tools to achieve this are open-source software, code documentation, and
    scientific vizualization, but I would like to expand this kit with new
    training on public speaking and writing.

    I envision, and contributed to design the TMS project as an opportunity to
    streamline my existing paleoglacier modelling workflow and extend its
    applications from selected regions to a global, fully automated, and
    multi-domain setup. While this is an optimistic plan for three years, I
    see it as a logical but ambitious continuation of my previous paleoglacier
    modelling work on the Cordilleran ice sheet and the Alps, building on my
    previous experience but also focusing on automating such studies instead of
    studying yet another region.

    Through increased attention to code documentation and visualization, I hope
    to get my work better noticed and understood outside academia (museum
    content creators, geographic tech startups, broader Python community),
    thereby opening new career opportunities. My ambition is that within three
    years, I will be able to provide academia and society with an automated,
    accessible tool that outlives my position here.


% ----------------------------------------------------------------------
\section{Project management}
% ----------------------------------------------------------------------

\guideline{
    Planning/development and implementation of research projects (e.g. a work
    package in a project)}

    Due to the interdisciplinary nature of our project, I expect a high degree
    of independence on my work. However, two Ph.D. students will be hired on
    the TMS grant from Jan. 2023, and depend on the availability and quality
    of my ice model output in order to inform their own biome connectivity and
    biodiversity modelling. We maintain regular communication between all
    project members and plan the following milestones on my work:

    \begin{itemize}
      \item Apr. 2023: technically valid ice model output, for a regional
        subset or lower resolution, that can be used by the Ph.D. students to
        develop their own workflow.
      \item Apr. 2024: scientifically valid output, publication-ready, that can
        be used by the Ph.D. students to prepare their own results.
      \item Apr. 2025: model output and methods shared in scientific
        publication, documented software, animations and interactive website.
    \end{itemize}

    To implement these goals, I plan to streamline my previously developed
    experience and \href{https://juseg.github.io/software}{software tools}
    into a single automated wrapper for the Parallel Ice Sheet Model
    (\href{https://pism.io}{PISM}). The foundations for this new tool are
    already laid as a fully documented Python plotting package that I began
    developing during my free time in the Spring of 2021
    (\href{hyoga.readthedocs.io}{Hyoga}). I will then apply for
    supercomputing resources and use the new framework to run PISM on multiple
    domains worldwide, produce ice sheet modelling results needed by the TMS
    project, and vizualisation products that can be used to communicate our
    methods and findings beyond academia.


% ----------------------------------------------------------------------
\section{Internationalization and network building}
% ----------------------------------------------------------------------

\guideline{
    Planned research stays abroad and other measures to increase national and
    international networks (attending conferences, hosting guest scientists
    etc.)}

    Networking with glacial geologists has been a continuous part of my work
    and remains critically needed here. Because glacial geology has
    traditionally been regional, our global project requires a global network.
    Later this year (29 Aug. - 2 Sep. 2022) we will invite two glacial
    geologists in Bergen to kick-off this network.

    \begin{itemize}
      \item Our new colleague Anna Hughes (University of Manchester, UK), as
        author of the latest reconstruction of Eurasian ice sheet, setting a
        new standard for ice-sheet reconstructions.
      \item My long-time colleague Martin Margold (Charles University, Czech
        Republic), who has conducted geological fieldwork and paleoglacier
        reconstructions in various parts of the world.
    \end{itemize}

    I am planning to attend paleoglaciology and glacial geology conferences
    where I can keep myself informed about latest developments in the field.
    This could include for instance the Nordic Geological Winter Meeting
    (\href{https://www.geologi.no/eventkalender/konferansekalender/vk23}{NGWM}
    4 – 6 Jan. 2023, Trondheim) or the International Union for Quaternary
    Research (\href{https://inquaroma2023.org}{INQUA} 14 – 20 July 2023, Roma).
    I am also closely following projects with different scientific objectives
    but similar vision on software communication and outreach. This is my
    motivation to participate in the upcoming
    \href{https://oggm.org/2022/05/09/6th-workshop-announcement/}{OGGM - PyGEM global glacier modelling workshop}
    (16 – 21 Sept., Finse).

    In the later stages of the project, I think that our results could attract
    much local interest from the academic, culture and education sectors if
    properly communicated. This could include planning longer visits at foreign
    institutions, organizing workshops, and producing outreach material in
    local languages.


% ----------------------------------------------------------------------
\section{Project proposal preparation}
% ----------------------------------------------------------------------

    \guideline{
        May be addressed at a later date. Participation in future project
        proposals and the possibility of developing own research ideas?}

    I am a member of and will seek additional funding from the
    \href{https://bjerknes.uib.no}{Bjerknes Centre for Climate Research} for
    short (few months) project extensions and inviting researchers from abroad
    (unfortunately our first try was just rejected), as well as from the
    \href{https://meltzerfondet.w.uib.no}{Meltzer Research Fund} for travels to
    conferences etc. I also plan to apply to the National Geographic Society
    \href{https://www.nationalgeographic.org/society/grants-and-investments/}{grants}
    which would be a good opportunity to increase our outreach in formerly
    glaciated, mountainous areas targeted in the numerical modelling work.

    We also consider to extend the duration and scope of the project through a
    new scientific proposal to e.g. the Norwegian Research Council or TMS.
    While we do not yet have concise plans, preliminary ideas for a
    glacier-centric project include:

    \begin{itemize}
      \item expanding to small valley glaciers, perhaps better resolved by
        a flowline model such as the Open Global Glacier Model (OGGM); or
      \item expanding to continental ice sheets (i.e. Eurasian, North American
        and Antarctic), more challenging due to their longer response time and
        interaction with the ocean.
    \end{itemize}


% ----------------------------------------------------------------------
\section{Project reporting}
% ----------------------------------------------------------------------

\guideline{
    (can also include financial management of research projects)}

    Suzette leads financial management and yearly project reporting on the TMS
    grant. I will contribute with reporting my own progress on the glacier
    modelling part.


% ----------------------------------------------------------------------
\section{Appraisal interview}
% ----------------------------------------------------------------------

\guideline{
    (\href{https://cp.compendia.no/universitetet-i-bergen/lederhandbok/70154\#openstep=70159}{template}
    (``\href{https://cp.compendia.no/universitetet-i-bergen/lederhandbok/?key=a69b52bc5b16c9e95456d369b96b9355}{Lederhåndboken}'',
    ``Medarbeiderutvikling'') (available in Norwegian only))
    }

    \textbf{a. Responsible for implementing appraisal interviews with the post
            doctor}

    John-Arvid Grytnes

    \textbf{b. Frequency (at least once per year)}

    Once per year.

    \textbf{c. Employee appraisal interviews will include monitoring the
            progress of the post doctor’s development plan, as well as
            discussing future career goals inside and outside academia (see 4
            above).}


% ----------------------------------------------------------------------
\section{Publication}
% ----------------------------------------------------------------------

    \guideline{
        \begin{itemize}
          \item[a.] Expected scientific publication.
          \item[b.] ``Author level''.
          \item[c.] Need course in writing? Expected scientific publications in
            relation to the scientific achievements, expected timing,
            suggestion for relevant publication channels etc.
        \end{itemize}}

    The primary goal of the project is global mountain paleoglacier simulations
    covering the last glacial cycle (120\,000 years). The results will be
    presented in a dedicated publication which will most likely be extensive
    and difficult due to necessary shortcomings on the climate inputs and
    probably innacurate model output when compared with detailed local
    knowledge of glacier history from geomorphology.

    Constant-climate simulations would be a logical first step yielding a first
    glimpse into the variety of glacier model responses to simplistic
    perturbations for different mountain ranges. If time allows, these results
    could be presented in a separate publication and reused in pedagogic
    outreach material around the question: ``how do glaciers build up to a
    given temperature anomaly, anywhere on Earth''.

    I do not think I need a course in academic writing. However I would be up
    for training for science communication or advanced English writing.


% ----------------------------------------------------------------------
\section{Communication/dissemination}
% ----------------------------------------------------------------------

    \guideline{
        \begin{itemize}
          \item[a.] Planned measures to communicate research and its results to
            the interested public (newspaper articles, contributions to
            ``Climate Snack'', contributions to events such as open science
            days like ``forskningsdagene'').
          \item[b.] Training in presentation techniques, communications, etc.
        \end{itemize}}

    Besides traditional academic publishing we plan the following output.

    \begin{itemize}
      \item The open-source, documented Python package for multi-domain
        paleoglacier modelling ``\href{hyoga.readthedocs.io}{Hyoga}, starting
        with the plotting interface (partly implemented in 2021), then input
        file preparation (topography and paleoclimate), and possibly a job
        wrapper (automatic job partitioning, queuing and restart on SLURM).
      \item \href{https://juseg.github.io/outreach/}{Glacier animations}
        for all mountain ranges worldwide for teachers and museums.
      \item An interactive website allowing users to explore modelling
        results in some sort of interactive way, such as zooming in and out on
        specific mountain ranges, and exploring their paleoglacier response to
        different temperature anomalies.
      \item Press release in English and Norwegian with help from the Bjerknes
        Centre where I understand there is good support for communicating our
        results to a broader audience.
      \item An article presenting our results in
        \href{https://theconversation.com}{The conversation}.
      \item A kids-reviewed article on paleoglaciers in
        \href{https://kids.frontiersin.org/}{Frontiers for Young minds}.
      \item Occasional progress updates in English on the
        \href{https://mountainsinmotion.w.uib.no/}{project website} and on my
        \href{juseg.github.io}{personal blog}.
    \end{itemize}


% ----------------------------------------------------------------------
\section{Leadership and innovation}
% ----------------------------------------------------------------------

    \guideline{
        \begin{itemize}
          \item[a.] Participation in boards and committees – scientific and
            administrative leadership
          \item[b.] Organizing scientific conferences/seminars
          \item[c.] Will the scientific work lead to innovation and attract
            interest beyond the actual project work?
        \end{itemize}}

    A most critical part of this project is the general sparsity and lack of
    organization of paleoglacier extent and dating data, which will be
    necessary to validate the model results. Regional reconstructions exist but
    rarely expand to more than a single ice sheet or mountain range. I believe
    this is currently a major barrier to progress in paleoglacier research, and
    hope that our efforts will provide upcoming generations of paleoglacier
    researchers with tools and inspiration to more systematically share and
    organize their data.

    While the topic of paleoglaciers is not directly related to the current
    demise of glaciers worldwide, I hope that our scientific and communication
    work will get as many people
    as possible excited in glacial geology, glaciers and biodiversity, and
    thereby invite them to explore and understand our world's changing nature.


% ----------------------------------------------------------------------
\section{Progress plan}
% ----------------------------------------------------------------------

    \guideline{
        Can be provided in form of a Gantt Chart, including the major elements
        of the development plan over the entire post doctor period. See
        example:}

    \begin{itemize}
      \item Apr. 2022 -- Sep. 2022: admin, networking, coding, applying for
        computing time.
      \item Sep. 2022 -- Apr. 2023: kick-off meeting, coding, multi-domain
        experimental runs.
      \item Apr. 2023 -- Apr. 2024: global experimental runs, production runs.
      \item Apr. 2024 -- Apr. 2025: paper writing, outreach.
    \end{itemize}

% ----------------------------------------------------------------------
\section{Signatures}
% ----------------------------------------------------------------------

    Postdoctoral research fellow (date, signature):

    Supervisor (date, signature):

    Approved by Head of Department (date, signature):


% ======================================================================
\end{document}
% ======================================================================
