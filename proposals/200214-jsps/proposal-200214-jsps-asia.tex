% Copyright (c) 2019--2020, Julien Seguinot <seguinot@vaw.baug.ethz.ch>
% Creative Commons Attribution-ShareAlike 4.0 International License
% (CC BY-SA 4.0, http://creativecommons.org/licenses/by-sa/4.0/)

% JSPS East Asian glaciation proposal due 2020.02.14.

\documentclass{article}

\usepackage{doi}
\usepackage{array}
\usepackage[T1]{fontenc}
\usepackage[utf8]{inputenc}
\usepackage[pdftex]{xcolor}
\usepackage[pdftex]{graphicx}
\usepackage[authoryear,round]{natbib}
\usepackage{bibentry}

% review mode
\usepackage{geometry}
%\usepackage{lineno}
%\linenumbers
%\linespread{1.5}

\graphicspath{{../../figures/}}

\definecolor{c0}{HTML}{1f77b4}
\definecolor{c1}{HTML}{ff7f0e}
\definecolor{c2}{HTML}{2ca02c}
\definecolor{c3}{HTML}{d62728}
\definecolor{c4}{HTML}{9467bd}
\definecolor{c5}{HTML}{8c564b}
\definecolor{c6}{HTML}{e377c2}
\definecolor{c7}{HTML}{7f7f7f}
\definecolor{c8}{HTML}{bcbd22}
\definecolor{c9}{HTML}{17becf}

\newcommand{\idea}[1]{\textcolor{c2}{\emph{[\textbf{IDEA:} #1]}}}
\newcommand{\note}[1]{\textcolor{c0}{\emph{[\textbf{NOTE:} #1]}}}
\newcommand{\todo}[1]{\textcolor{c3}{\emph{[\textbf{TODO:} #1]}}}

\hypersetup{colorlinks, citecolor=c0, linkcolor=c1, urlcolor=c6}

\title{Modelling the last glaciations in Northeast Asia and Japan}
\author{Julien Seguinot and Ayako Abe-Ouchi}
%\date{due February 14, 2020}


% ======================================================================
\begin{document}
% ======================================================================

\maketitle

%  1. Full Name
%  2. Nationality
%  3. Date of Birth
%  4. Sex (Put X in box below.)
%  5. Current Appointment
%  6. Academic Degree (Put X in box below and fill in the blanks.)
%  7. JSPS Fellowship(s) you were awarded in the past (Put X in box(s) below and fill in the blanks.)
%  8. Names of other Fellowship(s) that you are applying (Put X in box(s) below and fill in the blanks.)
%  9. Contact Information (Put an X in the box where you want to receive your award package from JSPS if you are selected, and fill in the blanks.)
% 10. Proposed Host Researcher/Host Institution
% 11. Higher Education (Start from the latest one. Include your current status if you are a doctoral student.)
% 12. Previous Appointments (Start from the latest one. Include your current appointment.)
% 13. Awards (Title, Organization, Year)
% 14. Language Ability
% 15. Past/Present Stay(s) in Japan over 3 months

% ----------------------------------------------------------------------
\setcounter{section}{15}
\section{Research Achievements and Results}
% ----------------------------------------------------------------------

    \emph{(Write concisely in a way that can be easily understood by persons
          outside your field of specialization, 1 page)}

% -- -- -- -- -- -- -- -- -- -- -- -- -- -- -- -- -- -- -- -- -- -- -- -
\paragraph{a. Research experience}
% -- -- -- -- -- -- -- -- -- -- -- -- -- -- -- -- -- -- -- -- -- -- -- -

    My primary field of research is paleo-glaciology, the study of past
    glacier evolution and its relationships to climate, geology and landscape
    building. More practically, I am using a numerical model, the Parallel Ice
    Sheet Model (PISM), based on ice physics, empirical knowledge from
    laboratory and field experiments, and mathematical computing techniques,
    to reconstruct the evolution of glaciers and ice sheets in the past. By
    calibrating the model to available geologic evidence, and by using some
    of the worlds fastest computers, I have used PISM to produce detailed
    reconstructions of past glacier evolution in the North American Cordillera
    \citep{Seguinot.etal.2014, Seguinot.etal.2016} and in the European Alps
    \citep{Seguinot.etal.2018}, and advised students and colleagues on similar
    work \citep{Becker.etal.2016, Jouvet.etal.2017a, Imhof.etal.2019}.

    I have also contributed to the field of glacier modelling at a more
    fundamental level by proposing enhancements of empirical glacier surface
    melt models \citep{Seguinot.2013, Seguinot.Rogozhina.2014}, and by
    participating in a subglacial hydrology model intercomparison
    \citep{Fleurian.etal.2018}. More recently, I have analysed field
    observations from ice boreholes drilled on a marine-terminating glacier
    in Greenland \citep{Seguinot.etal.Inreview}, a project which involved a
    one-year visit at Hokkaido University in Sapporo.

% -- -- -- -- -- -- -- -- -- -- -- -- -- -- -- -- -- -- -- -- -- -- -- -
\paragraph{b. Field experience}
% -- -- -- -- -- -- -- -- -- -- -- -- -- -- -- -- -- -- -- -- -- -- -- -

    During my Ph.D project on the North American Cordillera, I participated in
    two field campaigns in Western Canada, the second of which I obtained the
    funds for, and co-organised with another Ph.D student (M.~Margold). We
    sampled glacier-transported boulders for cosmogenic exposure dating of
    Cordilleran ice sheet deglaciation \citep{Menounos.etal.2017}.

    During my post-doc contract, I have participated in two fieldwork campaigns
    in Northwest Greenland jointly organised by Swiss (M.~Funk) and Japanese
    (S.~Sugiyama) teams to closely monitor one of the many fast-flowing and
    heavily-crevassed marine-terminating in Greenland, most of which have
    recently experienced acceleration and thinning due to the warming climate.
    I have participated the maintenance of automated cameras, seismometres
    (project of E.~Podolskiy), drone photogrametry control points
    \citep{Jouvet.etal.2017, Jouvet.etal.2018} and borehole instruments
    \citep{Seguinot.etal.Inreview}.

% -- -- -- -- -- -- -- -- -- -- -- -- -- -- -- -- -- -- -- -- -- -- -- -
\paragraph{c. Software and data}
% -- -- -- -- -- -- -- -- -- -- -- -- -- -- -- -- -- -- -- -- -- -- -- -

    Methods involved in my research are typically too complex to be published
    on paper. Therefore I believe that sharing software and data is essential
    to reproducibility and am committed to sharing them with other academics
    and with the rest of the society.

    I have developed a number of open-access tools for data processing and
    fieldwork planning (\url{https://github.com/juseg}), including an empirical
    glacier surface melt model (PyPDD), an automated workflow for satellite
    imagery (Sentinelflow), various plotting tools and PISM pre and
    post-processing tools. Through bug reporting and code writing I have also
    contributed to larger-scale software projects developed by international
    teams such as PISM (\url{https://github.com/pism/pism}) and Xarray
    (\url{https://github.com/pydata/xarray}). Model output for my simulations is
    archived in long-term public repositories and has regularly been downloaded
    for reuse (\url{https://zenodo.org/search?q=seguinot}).

% -- -- -- -- -- -- -- -- -- -- -- -- -- -- -- -- -- -- -- -- -- -- -- -
\paragraph{d. Education and outreach}
% -- -- -- -- -- -- -- -- -- -- -- -- -- -- -- -- -- -- -- -- -- -- -- -

    %\begin{itemize}
    %  \item{The results were reported by major media outlets in Switzerland and abroad}
    %  \item{Outreach materials are part of a permanent exhibition}
    %\end{itemize}


% ----------------------------------------------------------------------
\section{List of Major Publications}
% ----------------------------------------------------------------------

    \emph{(Authors (all), title, Journal,  Vol., No, pp.   -   , Month, Year,
          this list is to include your peer-reviewed papers that have been printed or
          are accepted for publication., 1 page)}

% -- -- -- -- -- -- -- -- -- -- -- -- -- -- -- -- -- -- -- -- -- -- -- -
\paragraph{First author peer-reviewed}
% -- -- -- -- -- -- -- -- -- -- -- -- -- -- -- -- -- -- -- -- -- -- -- -

    \bibliographystyle{abbrvnat}
    \nobibliography{../../../references/references.bib}

    \begin{itemize}
      \item\bibentry{Seguinot.etal.Inreview}
      \item\bibentry{Seguinot.etal.2018}
      \item\bibentry{Seguinot.etal.2016}
      \item\bibentry{Seguinot.etal.2014}
      \item\bibentry{Seguinot.Rogozhina.2014}
      \item\bibentry{Seguinot.2013}
    \end{itemize}


% -- -- -- -- -- -- -- -- -- -- -- -- -- -- -- -- -- -- -- -- -- -- -- -
\paragraph{Co-author peer-reviewed}
% -- -- -- -- -- -- -- -- -- -- -- -- -- -- -- -- -- -- -- -- -- -- -- -

    \begin{itemize}
      \item\bibentry{Imhof.etal.2019}
      \item\bibentry{Fleurian.etal.2018}
      \item\bibentry{Jouvet.etal.2018}
      \item\bibentry{Menounos.etal.2017}
      \item\bibentry{Jouvet.etal.2017a}
      \item\bibentry{Jouvet.etal.2017}
      \item\bibentry{Becker.etal.2016}
      \item\bibentry{Petit.etal.2009}
    \end{itemize}


% ----------------------------------------------------------------------
\section{Research plan in Japan}
% ----------------------------------------------------------------------

    \emph{(Follow plan below, 2 pages)}

% -- -- -- -- -- -- -- -- -- -- -- -- -- -- -- -- -- -- -- -- -- -- -- -
\paragraph{a. Background of proposed research plan}
% -- -- -- -- -- -- -- -- -- -- -- -- -- -- -- -- -- -- -- -- -- -- -- -

    Glaciers, by flowing and sliding across their beds, have reshaped the
    surface of the Earth to create some of its most impressive landscapes. For
    nearly three hundred years, explorers and scientists have learned to read
    the traces left by glaciers on the landscape and to understand past climate
    changes governing their fluctuations \citep[e.g.,][]{Venetz.1821}.
    In more recent decades, continuous paleoclimate records from deep sea
    sediments and ice cores have complemented the glacial landscape record to
    indicate that, for the last few million years, Earth has undergone tens of
    glacial cycles \citep{Lisiecki.Raymo.2005}, characterised by global sea
    level drops up to ca.~130\,m and massive ice build-up in presently
    unglacierized regions in North America, Scandinavia, Antarctica and
    Greenland \citep{Ehlers.etal.2011}.
    However, the glacial landscape record is sparse in time and space, such
    that much of the older evidence has been overprinted by subsequent
    glaciations.  In fact, much of the traces left by glaciers on the landscape
    date from the last major phase of global glacier expansion, the Last
    Glacial Maximum \citep[LGM,][]{Heyman.etal.2011}, also referred to Marine
    Oxygen Isotope Stage 2 (MIS~2) and dating from 25 (Alps, Laurentides) to 17
    (North American Cordillera) thousand years before the present (ka\,BP).

    A notable exception to this clustering of ages lie in the understudied
    glaciated mountain ranges of Northeast Asia and Alaska, where limited but
    growing evidence indicates that MIS~4 (71--57\,ka) glaciations were more
    expansive \citep{Batchelor.etal.2019}. This extends to the Japanese
    mountain ranges, where traces of glaciation are faint but were intensely
    debated following the discovery of glacier-striated rocks in the early
    twentieth century \citep{Yamazaki.1902}. While the application of modern
    dating techniques to Japan's paleo-glaciers has been limited, the available
    evidence indicates a probable MIS~4 maximum \citep{Sawagaki.Aoki.2011}.


% -- -- -- -- -- -- -- -- -- -- -- -- -- -- -- -- -- -- -- -- -- -- -- -
\paragraph{b. Purpose of proposed research}
% -- -- -- -- -- -- -- -- -- -- -- -- -- -- -- -- -- -- -- -- -- -- -- -

    We propose to attempt physics-based reconstructions of Japanese and
    Northeast Asian glaciations and aim to:

    \begin{itemize}
      \item{produce quantitative reconstructions of Japan mountain glaciers
            during MIS~2 and 4,}
      \item{estimate sea-level contribution of Northeast Asian mountains ice
            sheets during MIS~2 and 4,}
      \item{narrow the gap of knowledge on glaciation dynamics between
            eastern and western Arctic, and}
      \item{produce outreach materials to revive the interest on past glacier
            and climate changes in Japan.}
    \end{itemize}


% -- -- -- -- -- -- -- -- -- -- -- -- -- -- -- -- -- -- -- -- -- -- -- -
\paragraph{c. Proposed plan}
% -- -- -- -- -- -- -- -- -- -- -- -- -- -- -- -- -- -- -- -- -- -- -- -

    We plan to use the Parallel Ice Sheet Model (PISM), an open-source code
    embedding ice flow physics from continuum mechanics theory and empirical
    knowledge from laboratory experiments on ice and observations on modern
    glaciers. PISM is a mature code which has been developed and used by an
    international community to model the future evolution of the Greenland and
    Antarctic ice sheets, and the past glacier evolution including studies by
    the main proponent on the former Cordilleran and Alpine ice sheets. PISM
    takes full advantage of modern high-performance computers and computing
    resources will be applied for in separate proposals in Switzerland and
    Japan. Our proposed timeline is as follow:
    \begin{description}
      \item[Months 01--06:]
        Set-up PISM for multiple model domains in Japan (Fig.~\ref{fig:japan})
        and apply for computing resources. Gather available evidence from the
        English and Japanese language literature.
      \item[Months 07--12:]
        Run simulations of Japan mountain paleoglaciers, publish the results in
        scientific literature and prepare outreach materials and English and
        Japanese languages.
      \item[Months 13--18:]
        Set-up PISM for multiple model domains in Northeast Asia
        (Fig.~\ref{fig:asia}). Collaborate with museum and teachers on
        disseminating outreach materials.
      \item[Months 19--24:]
        Run simulations of Northeast Asia mountain ice sheets and publish the
        results in scientific literature.
    \end{description}


% -- -- -- -- -- -- -- -- -- -- -- -- -- -- -- -- -- -- -- -- -- -- -- -
\paragraph{d. Expected results and impacts}
% -- -- -- -- -- -- -- -- -- -- -- -- -- -- -- -- -- -- -- -- -- -- -- -

    Were the project funded we expect the following outcome:

    \begin{itemize}
      \item{Two publications summarizing the model results on Japan and
            Northeast Asia,}
      \item{A database of modelled glacier states for each stage and
            subregions,}
      \item{An open-source software framework for multi-domain paleoglacier
            modelling, and}
      \item{Outreach materials for schools and museums in Japan.}
    \end{itemize}

    Besides academic contributions, we hope that the project's outcome will
    revive the interest of Japanese researchers, students, and children to
    study past glacier changes, a fascinating topic at the interface of geology
    and climatology which has not received much recent attention in Japan
    despite its relevance to the urgent question of ongoing global climate
    change and its regional impacts.

    \begin{figure}
      \centerline{\includegraphics{asijap_domains}}
      \caption{%
        Documented glaciation (blue), planned model domains (red rectangles)
        and potentially glaciated active volanoes (red dots) in Japan.}
      \label{fig:japan}
    \end{figure}

    \begin{figure}
      \centerline{\includegraphics{asimis_domains}}
      \caption{%
        Documented MIS 2 (orange) and 4 (blue) glaciation and planned model
        domains in Northeast Asia and Alaska.}
      \label{fig:asia}
    \end{figure}

% -- -- -- -- -- -- -- -- -- -- -- -- -- -- -- -- -- -- -- -- -- -- -- -
\paragraph{e. References}
% -- -- -- -- -- -- -- -- -- -- -- -- -- -- -- -- -- -- -- -- -- -- -- -

    \begin{itemize}
      \item\bibentry{Batchelor.etal.2019}
      \item\bibentry{Ehlers.etal.2011}
      \item\bibentry{Heyman.etal.2011}
      \item\bibentry{Lisiecki.Raymo.2005}
      \item\bibentry{Sawagaki.Aoki.2011}
      \item\bibentry{Venetz.1821}
      \item\bibentry{Yamazaki.1902}
    \end{itemize}


% ----------------------------------------------------------------------
\section{Your Academic Goals and Career Prospects after the Fellowship}
% ----------------------------------------------------------------------

    \emph{(Ca. one third of a page)}

    My primary goal with this project is to foster Japan's young generations
    interest to exit the megacities and explore the beautiful landscapes
    surrounding, and thereby increase their awareness about our changing
    environment. By connecting past glacier changes in Japan to those of
    surrounding Asian regions, I hope to promote research collaboration between
    Japan and surrounding countries, particularly China where an active
    community of paleo-environmental researchers has been growing.

    By targeting a region characterised by a general anomaly in the timing of
    peak glaciation, I wish to develop my understanding of the atmospheric and
    oceanic mechanisms that may lead to asynchronous long-term changes on
    different sides of the planet. This is also reflected in my choice of host,
    Ayako Abe-Ouchi, one of the precursory researchers on glacier-climate
    coupling mechanisms and now one of the world's leading expert on
    paleoclimate research.

    Finally, this proposal fits within a longer-time project to develop an
    automated framework for paleoglacier modelling applicable anywhere on
    Earth. Within the next five years, I plan to streamline my previously
    developed experience and software tools (some already published
    open-source) into a single automated wrapper for PISM. After testing on
    multiple model domains in Northeast Asia and Japan, the resulting software
    tool would allow for both global-scale paleoglacier studies, and
    consistently set-up regional studies.


% ======================================================================
\end{document}
% ======================================================================
