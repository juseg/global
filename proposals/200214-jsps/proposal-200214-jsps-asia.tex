% Copyright (c) 2019--2020, Julien Seguinot <seguinot@vaw.baug.ethz.ch>
% Creative Commons Attribution-ShareAlike 4.0 International License
% (CC BY-SA 4.0, http://creativecommons.org/licenses/by-sa/4.0/)

% JSPS East Asian glaciation proposal due 2020.02.14.

\documentclass{article}

\usepackage{doi}
\usepackage[T1]{fontenc}
\usepackage[utf8]{inputenc}
\usepackage[pdftex]{xcolor}
\usepackage[pdftex]{graphicx}
\usepackage[authoryear,round]{natbib}
\usepackage{bibentry}

% review mode
\usepackage{geometry}
%\usepackage{lineno}
%\linenumbers
%\linespread{1.5}

\graphicspath{{../../figures/}}

\definecolor{c0}{HTML}{1f77b4}
\definecolor{c1}{HTML}{ff7f0e}
\definecolor{c2}{HTML}{2ca02c}
\definecolor{c3}{HTML}{d62728}
\definecolor{c4}{HTML}{9467bd}
\definecolor{c5}{HTML}{8c564b}
\definecolor{c6}{HTML}{e377c2}
\definecolor{c7}{HTML}{7f7f7f}
\definecolor{c8}{HTML}{bcbd22}
\definecolor{c9}{HTML}{17becf}

\newcommand{\idea}[1]{\textcolor{c2}{\emph{[\textbf{IDEA:} #1]}}}
\newcommand{\note}[1]{\textcolor{c0}{\emph{[\textbf{NOTE:} #1]}}}
\newcommand{\todo}[1]{\textcolor{c3}{\emph{[\textbf{TODO:} #1]}}}

\hypersetup{colorlinks, citecolor=c0, linkcolor=c1, urlcolor=c6}

\title{Modelling MIS4 glaciations in North Asia and Japan}
\author{Julien Seguinot and Ayako Abe-Ouchi}
%\date{due February 14, 2020}


% ======================================================================
\begin{document}
% ======================================================================

\maketitle

%  1. Full Name
%  2. Nationality
%  3. Date of Birth
%  4. Sex (Put X in box below.)
%  5. Current Appointment
%  6. Academic Degree (Put X in box below and fill in the blanks.)
%  7. JSPS Fellowship(s) you were awarded in the past (Put X in box(s) below and fill in the blanks.)
%  8. Names of other Fellowship(s) that you are applying (Put X in box(s) below and fill in the blanks.)
%  9. Contact Information (Put an X in the box where you want to receive your award package from JSPS if you are selected, and fill in the blanks.)
% 10. Proposed Host Researcher/Host Institution
% 11. Higher Education (Start from the latest one. Include your current status if you are a doctoral student.)
% 12. Previous Appointments (Start from the latest one. Include your current appointment.)
% 13. Awards (Title, Organization, Year)
% 14. Language Ability
% 15. Past/Present Stay(s) in Japan over 3 months

% ----------------------------------------------------------------------
\setcounter{section}{15}
\section{Research Achievements and Results}
% ----------------------------------------------------------------------

    \emph{(Write concisely in a way that can be easily understood by persons
          outside your field of specialization, 1 page)}

% alternative outline
% \paragraph{Research experience}
% \paragraph{Field experience}
% \paragraph{Software and data}
% \paragraph{Education and outreach}

% -- -- -- -- -- -- -- -- -- -- -- -- -- -- -- -- -- -- -- -- -- -- -- -
\paragraph{a. Alpine ice sheet}
% -- -- -- -- -- -- -- -- -- -- -- -- -- -- -- -- -- -- -- -- -- -- -- -

    \begin{itemize}
      \item{My primary field of research is paleo-ice sheet modelling}
      \item{Models that incorporate ice physics and use geological validation data}
      \item{My most recent publication is on the Alpine ice sheet}
      \item{The first complete pucture of the ice sheet through borders etc}
      \item{The ice sheet is more dynamic than previously thought}
      \item{The computations were performed on Piz Daint}
      \item{The results were reported by major media outlets in Switzerland and abroad}
      \item{Outreach materials are part of a permanent exhibition}
    \end{itemize}

% -- -- -- -- -- -- -- -- -- -- -- -- -- -- -- -- -- -- -- -- -- -- -- -
\paragraph{b. Bowdoin glacier}
% -- -- -- -- -- -- -- -- -- -- -- -- -- -- -- -- -- -- -- -- -- -- -- -

    \begin{itemize}
      \item{In 2015 and 2016 I have participated in two field campaigns}
      \item{I retrieved and analysed glacier borehole data}
      \item{I visited Hokkaido university for one year in 2018 -- 2019}
      \item{A first paper is in review}
    \end{itemize}

% -- -- -- -- -- -- -- -- -- -- -- -- -- -- -- -- -- -- -- -- -- -- -- -
\paragraph{c. Cordilleran ice sheet}
% -- -- -- -- -- -- -- -- -- -- -- -- -- -- -- -- -- -- -- -- -- -- -- -

    \begin{itemize}
      \item{My PhD thesis is about the Cordilleran ice sheet}
      \item{I applied PISM to paleoglaciology}
      \item{Due to complex climate high-resolution climate data is needed}
      \item{The ice sheet formed from numerous mountain ice fields}
      \item{Rapid retreat and ...}
    \end{itemize}

% -- -- -- -- -- -- -- -- -- -- -- -- -- -- -- -- -- -- -- -- -- -- -- -
\paragraph{d. Glacier melt parametrization}
% -- -- -- -- -- -- -- -- -- -- -- -- -- -- -- -- -- -- -- -- -- -- -- -

    \begin{itemize}
      \item{Day-to-day temperature variations significantly affect melt}
      \item{These are now implemented in PISM}
    \end{itemize}


% ----------------------------------------------------------------------
\section{List of Major Publications}
% ----------------------------------------------------------------------

    \emph{(Authors (all), title, Journal,  Vol., No, pp.   -   , Month, Year,
          this list is to include your peer-reviewed papers that have been printed or
          are accepted for publication., 1 page)}

% -- -- -- -- -- -- -- -- -- -- -- -- -- -- -- -- -- -- -- -- -- -- -- -
\paragraph{First author peer-reviewed}
% -- -- -- -- -- -- -- -- -- -- -- -- -- -- -- -- -- -- -- -- -- -- -- -

    \bibliographystyle{plainnat}
    \nobibliography{../../../references/references.bib}

    \begin{itemize}
      \item\bibentry{Seguinot.etal.Inreview}
      \item\bibentry{Seguinot.etal.2018}
      \item\bibentry{Seguinot.etal.2016}
      \item\bibentry{Seguinot.etal.2014}
      \item\bibentry{Seguinot.Rogozhina.2014}
      \item\bibentry{Seguinot.2013}
    \end{itemize}


% -- -- -- -- -- -- -- -- -- -- -- -- -- -- -- -- -- -- -- -- -- -- -- -
\paragraph{Co-author peer-reviewed}
% -- -- -- -- -- -- -- -- -- -- -- -- -- -- -- -- -- -- -- -- -- -- -- -

    \begin{itemize}
      \item\bibentry{Imhof.etal.2019}
      \item\bibentry{Fleurian.etal.2018}
      \item\bibentry{Jouvet.etal.2018}
      \item\bibentry{Menounos.etal.2017}
      \item\bibentry{Jouvet.etal.2017a}
      \item\bibentry{Jouvet.etal.2017}
      \item\bibentry{Becker.etal.2016}
      \item\bibentry{Petit.etal.2009}
    \end{itemize}


% ----------------------------------------------------------------------
\section{Research plan in Japan}
% ----------------------------------------------------------------------

    \emph{(Follow plan below, 2 pages)}

% -- -- -- -- -- -- -- -- -- -- -- -- -- -- -- -- -- -- -- -- -- -- -- -
\paragraph{a. Background of proposed research plan}
% -- -- -- -- -- -- -- -- -- -- -- -- -- -- -- -- -- -- -- -- -- -- -- -

    \begin{itemize}
      \item{Typically older glaciations evidence is overriden by the LGM}
      \item{The exception is east asia and alaska where MIS 4 > MIS 2}
      \item{Their glacial dynamics are understudied}
    \end{itemize}

% -- -- -- -- -- -- -- -- -- -- -- -- -- -- -- -- -- -- -- -- -- -- -- -
\paragraph{b. Purpose of proposed research}
% -- -- -- -- -- -- -- -- -- -- -- -- -- -- -- -- -- -- -- -- -- -- -- -

    \begin{itemize}
      \item{Fill the gap of knowledge of past glacier dyn in East vs West Arctic}
      \item{East Asia ice volume contribution to pre-LGM sea level drops}
    \end{itemize}

% -- -- -- -- -- -- -- -- -- -- -- -- -- -- -- -- -- -- -- -- -- -- -- -
\paragraph{c. Proposed plan}
% -- -- -- -- -- -- -- -- -- -- -- -- -- -- -- -- -- -- -- -- -- -- -- -

    \begin{itemize}
      \item{Apply for computing resources (CH/JP)}
      \item{Setup PISM for multiple domains in east Asia}
    \end{itemize}

% -- -- -- -- -- -- -- -- -- -- -- -- -- -- -- -- -- -- -- -- -- -- -- -
\paragraph{d. Expected results and impacts}
% -- -- -- -- -- -- -- -- -- -- -- -- -- -- -- -- -- -- -- -- -- -- -- -

    \begin{itemize}
      \item{A database of modelled glacier states}
      \item{An estimate of ice volume in ne asia through mis stages}
      \item{A framework for multidomain paleoglacier modelling}
    \end{itemize}

    \begin{figure}
      \centerline{\includegraphics{asijap_domains}}
      \caption{%
        Documented glaciation (blue), planned model domains (red rectangles)
        and potentially glaciated active volanoes (red dots) in Japan.}
      \label{fig:japan}
    \end{figure}

    \begin{figure}
      \centerline{\includegraphics{asimis_domains}}
      \caption{%
        Documented MIS 2 (orange) and 4 (blue) glaciation and planned model
        domains in Northeast Asia and Alaska.}
      \label{fig:asia}
    \end{figure}


% ----------------------------------------------------------------------
\section{Your Academic Goals and Career Prospects after the Fellowship}
% ----------------------------------------------------------------------

    \emph{(Ca. one third of a page)}

    \begin{itemize}
      \item{Spark interest of Japanese students}
      \item{Extend to asia, panarctic and global}
      \item{PMIP output}
    \end{itemize}

% ======================================================================
\end{document}
% ======================================================================
