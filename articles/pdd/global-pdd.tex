% Copyright (c) 2024, Julien Seguinot (juseg.dev)
% Creative Commons Attribution-ShareAlike 4.0 International License
% (CC BY-SA 4.0, http://creativecommons.org/licenses/by-sa/4.0/)

% Global PDD paper
% ================

\documentclass[manuscript]{copernicus}

% custom colours (colorbrewer2.org)
\definecolor{Bu}{cmyk}{1.00,0.45,0.00,0.07}  % Blues
\definecolor{Gn}{cmyk}{1.00,0.20,1.00,0.00}  % Greens
\definecolor{Rd}{cmyk}{0.35,0.95,0.85,0.00}  % Reds
\definecolor{Or}{cmyk}{0.35,0.75,1.00,0.00}  % Oranges
\definecolor{Pu}{cmyk}{0.70,0.80,0.00,0.00}  % Purples
\definecolor{Br}{cmyk}{0.40,0.75,1.00,0.00}  % YlOrBr

% custom commands (remove before submission)
\newcommand{\note}[1]{\textcolor{Or}{\emph{[\textbf{NOTE:} #1]}}}
\newcommand{\todo}[1]{\textcolor{Rd}{\emph{[\textbf{TODO:} #1]}}}

% color links (remove before submission)
\hypersetup{colorlinks, citecolor=Bu, linkcolor=Bu, urlcolor=Bu}

% figures directory
\graphicspath{{../../figures/}}

% document properties
\title{Global glacial inception threshold from positive degree-days}
\Author[1]{Julien}{Seguinot}
\affil[1]{Department of Water and Climate, Vrije Universiteit Brussel, Brussels, Belgium}
\runningtitle{Last glacial cycle glacier erosion potential in the Alps}
\runningauthor{J.~Seguinot et al.}


% ======================================================================
\begin{document}
% ======================================================================

\maketitle

\begin{abstract}

    Glaciations of the Pleistocene have left a global imprint expanding from
    polar plains to equatorial mountains on all continents. This glacial record
    has been systematically researched for nearly two centuries. However, its
    rich diversity, as well as fieldwork logistics, numerical flow-modelling
    challenges and paleoclimate unknowns have often constrained paleoglacier
    studies to remain regional.
    %
    Here, we present a global map of temperature changes needed to incept
    glaciers on any given location. Using 30-arcsec horizontal resolution
    downscaled climatologies from CHELSA-2.1 and CHELSA-W5E5 data, and
    temperature anomalies ranging from +5 to -20\,K, a positive-degree-day snow
    accumulation and melt model was applied globally to compute the glacial
    inception threshold with a precision of 0.2\,K.
    %
    Our product reproduces many known glacier and ice-sheet inception centres,
    while also hinting at potentially undocumented or unproven glaciations.
    Our map may not capture all the complexity of glacial evidence reported in
    the literature, but we hope it may serve to hint at potential targets for
    future field and modelling studies, and provide a foundation towards global
    paleoglacier studies.

\end{abstract}


% ----------------------------------------------------------------------
\introduction
% ----------------------------------------------------------------------

% ----------------------------------------------------------------------
\section{Methods}
% ----------------------------------------------------------------------

% -- -- -- -- -- -- -- -- -- -- -- -- -- -- -- -- -- -- -- -- -- -- -- -
\subsection{Climatology}
% -- -- -- -- -- -- -- -- -- -- -- -- -- -- -- -- -- -- -- -- -- -- -- -

% -- -- -- -- -- -- -- -- -- -- -- -- -- -- -- -- -- -- -- -- -- -- -- -
\subsection{Surface mass balance}
% -- -- -- -- -- -- -- -- -- -- -- -- -- -- -- -- -- -- -- -- -- -- -- -

% -- -- -- -- -- -- -- -- -- -- -- -- -- -- -- -- -- -- -- -- -- -- -- -
\subsection{Glacial inception threshold}
% -- -- -- -- -- -- -- -- -- -- -- -- -- -- -- -- -- -- -- -- -- -- -- -

% ----------------------------------------------------------------------
\section{Results}
% ----------------------------------------------------------------------

% -- -- -- -- -- -- -- -- -- -- -- -- -- -- -- -- -- -- -- -- -- -- -- -
\subsection{Spatial distribution}
% -- -- -- -- -- -- -- -- -- -- -- -- -- -- -- -- -- -- -- -- -- -- -- -

    Fig. 1 -- Global inception threshold map.

    % \begin{figure*}
    %   \centerline{\includegraphics{glopdd_threshold}}
    %   \caption{%
    %     Global glacial inception threshold.
    %     % \textbf{(a)} Panel a.
    %     % \textbf{(b)} Panel b.
    %     }
    %     \label{fig:threshold}
    % \end{figure*}

% -- -- -- -- -- -- -- -- -- -- -- -- -- -- -- -- -- -- -- -- -- -- -- -
\subsection{Equilibrium line altitude}
% -- -- -- -- -- -- -- -- -- -- -- -- -- -- -- -- -- -- -- -- -- -- -- -

    Fig. 2 -- Elevation and altitude distribution.

% -- -- -- -- -- -- -- -- -- -- -- -- -- -- -- -- -- -- -- -- -- -- -- -
\subsection{Regional showcase}
% -- -- -- -- -- -- -- -- -- -- -- -- -- -- -- -- -- -- -- -- -- -- -- -

    Fig. 3 -- Examples ice sheet, mountain, mismatches.

% ----------------------------------------------------------------------
\section{Discussion}
% ----------------------------------------------------------------------

% -- -- -- -- -- -- -- -- -- -- -- -- -- -- -- -- -- -- -- -- -- -- -- -
\subsection{Sensitivity to climate data}
% -- -- -- -- -- -- -- -- -- -- -- -- -- -- -- -- -- -- -- -- -- -- -- -

    Fig. 4 -- Difference with CHELSA-ERA5.

% -- -- -- -- -- -- -- -- -- -- -- -- -- -- -- -- -- -- -- -- -- -- -- -
\subsection{Effect of reduced precipitation}
% -- -- -- -- -- -- -- -- -- -- -- -- -- -- -- -- -- -- -- -- -- -- -- -

    Fig. 5 -- Effect of paleo-precip reductions.

% -- -- -- -- -- -- -- -- -- -- -- -- -- -- -- -- -- -- -- -- -- -- -- -
\subsection{Comparison with glacial temperature}
% -- -- -- -- -- -- -- -- -- -- -- -- -- -- -- -- -- -- -- -- -- -- -- -

    Fig. 6 -- Ratio to an LGM temperature map.

% -- -- -- -- -- -- -- -- -- -- -- -- -- -- -- -- -- -- -- -- -- -- -- -
% \subsection{Subsection}
% \label{sec:subsection}
% -- -- -- -- -- -- -- -- -- -- -- -- -- -- -- -- -- -- -- -- -- -- -- -

    % References to \citep[e.g.,][for instance]{Koppes.etal.2015},
    % Fig.~\ref{fig:threshold}a, and Sect.~\ref{sec:subsection}. Let
    % %
    % \begin{equation}
    %     \dot{e} = K_\mathrm{g} u_\mathrm{b}^l ,
    % \end{equation}
    % %
    % where $K_g = 5.2\times 10^{-11}\,m^{1-l}\,a^{l-1}$. A list:
    % \begin{itemize}
    %   \item Conclusion.
    %   \item Conclusion.
    % \end{itemize}


% ----------------------------------------------------------------------
\conclusions
% ----------------------------------------------------------------------

% ----------------------------------------------------------------------
% Acknowledgements
% ----------------------------------------------------------------------

% \codedataavailability{%
%     The useful section.}

% \authorcontribution{%
%     We did it.}

% \competinginterests{%
%     The authors declare that they have no conflict of interest.}

% \begin{acknowledgements}
%     Thanks for all the frites.
% \end{acknowledgements}


% ----------------------------------------------------------------------
% References
% ----------------------------------------------------------------------

% \bibliographystyle{copernicus}
% \bibliography{../../../references/references}


% ======================================================================
\end{document}
% ======================================================================
